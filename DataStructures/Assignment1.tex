% !TEX TS-program = pdflatex
% !TEX encoding = UTF-8 Unicode

\documentclass[11pt]{article} % use larger type; default would be 10pt

\usepackage[utf8]{inputenc} % set input encoding (not needed with XeLaTeX)

%%% PAGE DIMENSIONS
\usepackage[top=0.6in, left=0.8in, right=0.8in, bottom=0.7in]{geometry} % to change the page dimensions
\geometry{a4paper} % or letterpaper (US) or a5paper or....
% \geometry{margins=2in} % for example, change the margins to 2 inches all round
% \geometry{landscape} % set up the page for landscape

\usepackage{graphicx} % support the \includegraphics command and options

\usepackage[parfill]{parskip} % Activate to begin paragraphs with an empty line rather than an indent

%%% PACKAGES
\usepackage{booktabs} % for much better looking tables
\usepackage{array} % for better arrays (eg matrices) in maths
%\usepackage{paralist} % very flexible & customisable lists (eg. enumerate/itemize, etc.)
\usepackage{verbatim} % adds environment for commenting out blocks of text & for better verbatim
\usepackage{subfig} % make it possible to include more than one captioned figure/table in a single float
\usepackage{mathtools} % for math environments like align
\usepackage{amssymb} % for symbols like \therefore

%%% OPTIONAL PACKAGES
%\usepackage{braket}

%%% HEADERS & FOOTERS
\usepackage{fancyhdr} % This should be set AFTER setting up the page geometry
\pagestyle{fancy} % options: empty , plain , fancy
\renewcommand{\headrulewidth}{0pt} % customise the layout...
\lhead{}\chead{}\rhead{}
\lfoot{}\cfoot{\thepage}\rfoot{}

%%% SECTION TITLE APPEARANCE
%\usepackage{sectsty}
%\allsectionsfont{\sffamily\mdseries\upshape} % (See the fntguide.pdf for font help)

%%% ToC (table of contents) APPEARANCE
%\usepackage[nottoc,notlof,notlot]{tocbibind} % Put the bibliography in the ToC
%\usepackage[titles,subfigure]{tocloft} % Alter the style of the Table of Contents
%\renewcommand{\cftsecfont}{\rmfamily\mdseries\upshape}
%\renewcommand{\cftsecpagefont}{\rmfamily\mdseries\upshape} % No bold!

%%% END Article customizations

\author{Josh Wainwright \\ UID:1079596}

\title{Data Stuctures: Assignment 1 }

\date{}

\begin{document}

\maketitle

% \subsection*{Wikipedia Articles}

\begin{enumerate}
	\item \textbf{Wikipedia Articles} --- \textit{Hash map.}

		\begin{itemize}
			\item Hash key is article title, hash value is the article text.
			\item Access times want to be fast for an article that is read more times
				than it is edited. For every access of an article, the title is used to
				locate the article which can be displayed, so the average and worst
				case times are equal, i.e.\ constant.
			\item When the article is edited, or a new article is added, the average
				case will still be constant. The worst case is when the hash map must
				be resized, in which case there will be a significant time increase.
		\end{itemize}

		% \subsection*{Driving Lessons}
	\item \textbf{Driving Lessons } --- \textit{Tree set.}

		\begin{itemize}
			\item Just need to individually store the name of each customer who has had
				a free lesson.
			\item Set allows no duplicates, so a user cannot be added more than once.
				If implementation allows, could check the customer's status in the set
				at the same time as adding them. When a customer tries to book a free
				lesson, try to add them to the set. If this fails, then they already
				exist so cannot have a free lesson. If it is successful, then they have
				their lesson, but if they try to book again, their name exists in the
				set.
			\item When checking a name, the worst case is that they are not already in
				the set, i.e.\ linear, same for adding.
		\end{itemize}

	\item \textbf{Painting Prices} --- \textit{}

	\item \textbf{Birth Cities} --- \textit{}

\end{enumerate}

\end{document}
