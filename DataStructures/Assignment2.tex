% !TEX TS-program = pdflatex
% !TEX encoding = UTF-8 Unicode

\documentclass[11pt]{article} % use larger type; default would be 10pt

\usepackage[utf8]{inputenc} % set input encoding (not needed with XeLaTeX)

%%% PAGE DIMENSIONS
\usepackage[top=0.6in, left=0.8in, right=0.8in, bottom=0.7in]{geometry} % to change the page dimensions
\geometry{a4paper} % or letterpaper (US) or a5paper or....
% \geometry{margins=2in} % for example, change the margins to 2 inches all round
% \geometry{landscape} % set up the page for landscape

\usepackage{graphicx} % support the \includegraphics command and options

\usepackage[parfill]{parskip} % Activate to begin paragraphs with an empty line rather than an indent

%%% PACKAGES
\usepackage{booktabs} % for much better looking tables
\usepackage{array} % for better arrays (eg matrices) in maths
%\usepackage{paralist} % very flexible & customisable lists (eg. enumerate/itemize, etc.)
\usepackage{verbatim} % adds environment for commenting out blocks of text & for better verbatim
\usepackage{subfig} % make it possible to include more than one captioned figure/table in a single float
\usepackage{mathtools} % for math environments like align
\usepackage{amssymb} % for symbols like \therefore

%%% OPTIONAL PACKAGES
%\usepackage{braket}

%%% HEADERS & FOOTERS
\usepackage{fancyhdr} % This should be set AFTER setting up the page geometry
\pagestyle{fancy} % options: empty , plain , fancy
\renewcommand{\headrulewidth}{0pt} % customise the layout...
\lhead{}\chead{}\rhead{}
\lfoot{}\cfoot{\thepage}\rfoot{}

%%% SECTION TITLE APPEARANCE
%\usepackage{sectsty}
%\allsectionsfont{\sffamily\mdseries\upshape} % (See the fntguide.pdf for font help)

%%% ToC (table of contents) APPEARANCE
%\usepackage[nottoc,notlof,notlot]{tocbibind} % Put the bibliography in the ToC
%\usepackage[titles,subfigure]{tocloft} % Alter the style of the Table of Contents
%\renewcommand{\cftsecfont}{\rmfamily\mdseries\upshape}
%\renewcommand{\cftsecpagefont}{\rmfamily\mdseries\upshape} % No bold!

%%% END Article customizations

\author{Josh Wainwright \\ UID:1079596}

\title{Data Stuctures: Assignment 2 }

\date{}

\begin{document}

\maketitle

\section{Heapsort}

Heap sort is an algorithm that sorts a set of elements using the heap ADT.  A
heap allows a set of elements to be arranged such that the element with the
highest priority is available directly with complexity O(1).

\begin{enumerate}
	\item Add all elements to be sorted onto an initially empty heap
	\item Remove each element ``in order''.
	\item Since the heap presents the highest priority element, simply by
		maintaining the validity of the heap, the next largest element will
		always be the root node.
	\item So to remove ``in order'', simply take off the root node and verify
		that the heap is valid.
\end{enumerate}

Adding elements to the heap involves the steps:
\begin{enumerate}
	\item Place the new element in the last position available on the final
		level of the tree structure. If the final row is full, start a new row
		from the left.
	\item Compare the new node with its parent. \label{stp:compare}
	\item If the parent has a higher priority (is larger), the element is in
		the correct position and the addition is complete.
	\item If the parent has a lower priority (is smaller) replace the parent
		with the newly added node. \label{stp:swap}
	\item Repeat steps~\ref{stp:compare}--\ref{stp:swap}.
\end{enumerate}

\section{Dijkstra's Shortest Path Algorithm}

Dijkstra's algorithm locates the shortest path from a starting node to any
given destination node across a graph. It does this by finding the shortest
path for every node in the graph.

A priority collection can be used when calculating the distance to the
neighbours of the elements that have already been considered. Since the
shortest path is wanted, the paths with the shortest distance have the highest
priority and so, when the algorithm has completed, the shortest path is found
by removing the nodes from the collection by priority. This will give the path
from the destination to the starting node with the shortest path. Thus, the
reverse of this path is the shortest path from start to destination.

\end{document}
