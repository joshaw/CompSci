% !TEX TS-program = pdflatex
% !TEX encoding = UTF-8 Unicode

\documentclass[11pt]{article} % use larger type; default would be 10pt

\usepackage[utf8]{inputenc} % set input encoding (not needed with XeLaTeX)

%%% PAGE DIMENSIONS
\usepackage[top=0.6in, left=0.8in, right=0.8in, bottom=0.7in]{geometry} % to change the page dimensions
\geometry{a4paper} % or letterpaper (US) or a5paper or....
% \geometry{margins=2in} % for example, change the margins to 2 inches all round
% \geometry{landscape} % set up the page for landscape

\usepackage{graphicx} % support the \includegraphics command and options

\usepackage[parfill]{parskip} % Activate to begin paragraphs with an empty line rather than an indent

%%% PACKAGES
\usepackage{booktabs} % for much better looking tables
\usepackage{array} % for better arrays (eg matrices) in maths
%\usepackage{paralist} % very flexible & customisable lists (eg. enumerate/itemize, etc.)
\usepackage{verbatim} % adds environment for commenting out blocks of text & for better verbatim
\usepackage{subfig} % make it possible to include more than one captioned figure/table in a single float
\usepackage{mathtools} % for math environments like align
\usepackage{amssymb} % for symbols like \therefore

%%% OPTIONAL PACKAGES
%\usepackage{braket}

%%% HEADERS & FOOTERS
\usepackage{fancyhdr} % This should be set AFTER setting up the page geometry
\pagestyle{fancy} % options: empty , plain , fancy
\renewcommand{\headrulewidth}{0pt} % customise the layout...
\lhead{}\chead{}\rhead{}
\lfoot{}\cfoot{\thepage}\rfoot{}

%%% SECTION TITLE APPEARANCE
%\usepackage{sectsty}
%\allsectionsfont{\sffamily\mdseries\upshape} % (See the fntguide.pdf for font help)

%%% ToC (table of contents) APPEARANCE
%\usepackage[nottoc,notlof,notlot]{tocbibind} % Put the bibliography in the ToC
%\usepackage[titles,subfigure]{tocloft} % Alter the style of the Table of Contents
%\renewcommand{\cftsecfont}{\rmfamily\mdseries\upshape}
%\renewcommand{\cftsecpagefont}{\rmfamily\mdseries\upshape} % No bold!

%%% END Article customizations

\author{Josh Wainwright \\ UID:1079596}

\title{Exercise Sheet 5 }

\date{}

\begin{document}

\maketitle
%\tableofcontents
%\vspace{1cm}\hrule \vspace{1cm}
%\newpage

\section{Diffie-Hellman Key Exchange}
\begin{equation}
	p=23,\, g=3,\, a=17,\, b=11
\end{equation}
\begin{enumerate}
	\item Alice and Bob agree (publicly) on a prime number, $p=23$ (ie on
		$\mathbb{Z}^{*}_{p}$) and on an element $g \in \mathbb{Z}^{*}_{p}$ that
		generates the finite group  $G_q$.
	\item Alice picks a random positive integer $a=17$ and sends $g^a = 3^{17}
		= 129140163$ to Bob.
	\item Bob picks a random positive integer $b=11$ and sends $g^a = 3^{11} =
		177147$ to Alice.
	\item Alice computes the key $K = {(g^b)}^a = {(177147)}^{17}$.
	\item Bob computes the key $K = {(g^b)}^a = {(129140163)}^{11}$.
	\item Both Alice and Bob now have the same key, $K =
		1\-6\-6\-5\-9\-9\-8\-6\-0\-1\-7\-6\-3\-0\-9\-8\-0\-0\-0\-4\-6\-0\-2\-6\-6\-3\-4\-5\-2\-4\-9\-8\-6\-2\-3\-3\-5\-4\-8\-1\-7\-9\-0\-4\-0\-6\-6\-7\-0\-3\-8\-2\-9\-5\-2\-3\-5\-3\-9\-1\-4\-2\-4\-2\-8\-4\-2\-2\-1\-2\-5\-9\-3\-6\-9\-2\-1\-3\-2\-5\-4\-1\-4\-8\-8\-6\-7\-3\-8\-7$
\end{enumerate}

\section{ElGammel}
\begin{align}
	p=47,\, q=23,\, g=2,\, \Rightarrow G_{23} = \langle 2 \rangle \\
	x=6,\, y=10,\, M=9
\end{align}
\subsection{Key Generation}
\begin{enumerate}
	\item Generate primes $p=47$ and $q=23$ as well as an element $g \in
		\mathbb{Z}^{*}_{p}$ that generates the subgroup $G_{q} = G_{23} =
		\langle 2 \rangle$.
	\item Choose a random $x=6$ from $\{0, \ldots, 22\}$.
	\item Compute $h=g^x \mod p = 2^6 \mod 47 = 17$.
	\item Publish the public key $\hat{K} = (G_q, q, g, h) =
		(\{1,2,4,8,16,9,18, 13, 3, 6, 12\}, 23, 23, 2, 17)$.
	\item Retain the private key $K=x$.
\end{enumerate}

\subsection{Encryption}
\begin{enumerate}
	\item The message, $M$, is considered to be an element of $G$.
	\item Choose a random $y=10$ from $\{0,\ldots, 22\}$ then calculate $c_1 =
		g^y = 1024$ and $c_2 = M \cdot h^y = 10376293541461622784$.
	\item The ciphertext is then $C = (c_1, c_2) = (1024, 10376293541461622784)$.
\end{enumerate}

\subsection{Decryption}
\begin{enumerate}
	\item Using the secret key $K=x=6$,
	\item Compute
		\begin{align*}
			M &= c_2 \cdot c{_1}^{-x} \\
			  &= c_2 \cdot {(c{_1}^{-1})}^x \\
			  &= 10376293541461622784 \times {(1024^{-1})}^6 \\
			  &= 9 \\
		\end{align*}
\end{enumerate}

\section{RSA Encryption}
\begin{equation}
	p=7,\, q=11,\, e=17,\, M=48
\end{equation}
\subsection{Key Generation}
\begin{enumerate}
	\item Generate two random primes $p=7$ and $q=17$.
	\item Compute $n=pq = 7 \times 17 = 119$ and $\phi = (p-1)(q-1) = 6 \times
		16 = 96$.
	\item Select a random integer $e=17$, $1<e<\phi$ such that
		$\text{gcd}(e,\phi) = 1$.
	\item Use the extended Euclidean algorithm to compute the unique integer
		$d=17$, $1<d<\phi$, such that $\text{e}\times\text{d} \equiv 1 \mod
		(\phi)$.
	\item Alice's public key is $(n,e) = (119, 17)$, and Alice's private key is $d=17$
\end{enumerate}

\subsection{Encryption}
\begin{enumerate}
	\item Bob obtains Alice's public key $(n,e) = (119,17)$.
	\item Bobs computes the ciphertext, $c$,
		\begin{align*}
			c &= M^e \mod n \\
			  &= 48^{17} \mod 119 = 48.
		\end{align*}
		and send $c$ to Alice.
\end{enumerate}

\subsection{Decryption}
\begin{enumerate}
	\item Alice receives the ciphertext, $c$, from Bob and recovers the
		plaintext, $M = c^d \mod n$, with the private key, $d$,
		\begin{align*}
			M &= c^d \mod n \\
			  &= 48^{17} \mod 119 = 48.
		\end{align*}
\end{enumerate}

\section{Invertible Elements of $\mathbb{Z}_{34}$}
The members of $\left(\mathbb{Z}_{34}^*, \cdot\right)$ is not a group, but we
can find the members of this set that have multiplicative inverses, elements $x$
for which an element $y$ exists such that $xy \equiv 1(\mod 34)$.

We can disregard all the even elements since these all share a divisor with 34,
namely 2, as well as the element 17. We know that there must be 16 invertable
elements. The remaining elements of the multiplication table modulo 34 is shown
below.

\begin{tabular}{rllllllllrrrrrrrrr}

\multicolumn{1}{l}{\textbf{}} & \multicolumn{1}{r}{\textbf{1}} & \multicolumn{1}{r}{\textbf{3}} & \multicolumn{1}{r}{\textbf{5}} & \multicolumn{1}{r}{\textbf{7}} & \multicolumn{1}{r}{\textbf{9}} & \multicolumn{1}{r}{\textbf{11}} & \multicolumn{1}{r}{\textbf{13}} & \multicolumn{1}{r}{\textbf{15}} & \textbf{17} & \textbf{19} & \textbf{21} & \textbf{23} & \textbf{25} & \textbf{27} & \textbf{29} & \textbf{31} & \textbf{33} \\ 
\textbf{1} & \multicolumn{1}{r}{\textbf{\textit{1}}} & \multicolumn{1}{r}{3} & \multicolumn{1}{r}{5} & \multicolumn{1}{r}{7} & \multicolumn{1}{r}{9} & \multicolumn{1}{r}{11} & \multicolumn{1}{r}{13} & \multicolumn{1}{r}{15} & 17 & 19 & 21 & 23 & 25 & 27 & 29 & 31 & 33 \\ 
\textbf{3} &  & \multicolumn{1}{r}{9} & \multicolumn{1}{r}{15} & \multicolumn{1}{r}{21} & \multicolumn{1}{r}{27} & \multicolumn{1}{r}{33} & \multicolumn{1}{r}{5} & \multicolumn{1}{r}{11} & 17 & 23 & 29 & \textbf{\textit{1}} & 7 & 13 & 19 & 25 & 31 \\ 
\textbf{5} &  &  & \multicolumn{1}{r}{25} & \multicolumn{1}{r}{\textbf{\textit{1}}} & \multicolumn{1}{r}{11} & \multicolumn{1}{r}{21} & \multicolumn{1}{r}{31} & \multicolumn{1}{r}{7} & 17 & 27 & 3 & 13 & 23 & 33 & 9 & 19 & 29 \\ 
\textbf{7} &  &  &  & \multicolumn{1}{r}{15} & \multicolumn{1}{r}{29} & \multicolumn{1}{r}{9} & \multicolumn{1}{r}{23} & \multicolumn{1}{r}{3} & 17 & 31 & 11 & 25 & 5 & 19 & 33 & 13 & 27 \\ 
\textbf{9} &  &  &  &  & \multicolumn{1}{r}{13} & \multicolumn{1}{r}{31} & \multicolumn{1}{r}{15} & \multicolumn{1}{r}{33} & 17 & \textbf{\textit{1}} & 19 & 3 & 21 & 5 & 23 & 7 & 25 \\ 
\textbf{11} &  &  &  &  &  & \multicolumn{1}{r}{19} & \multicolumn{1}{r}{7} & \multicolumn{1}{r}{29} & 17 & 5 & 27 & 15 & 3 & 25 & 13 & \textbf{\textit{1}} & 23 \\ 
\textbf{13} &  &  &  &  &  &  & \multicolumn{1}{r}{33} & \multicolumn{1}{r}{25} & 17 & 9 & \textbf{\textit{1}} & 27 & 19 & 11 & 3 & 29 & 21 \\ 
\textbf{15} &  &  &  &  &  &  &  & \multicolumn{1}{r}{21} & 17 & 13 & 9 & 5 & \textbf{\textit{1}} & 31 & 27 & 23 & 19 \\ 
\textbf{17} &  &  &  &  &  &  &  &  & 17 & 17 & 17 & 17 & 17 & 17 & 17 & 17 & 17 \\ 
\textbf{19} &  &  &  &  &  &  &  &  & \multicolumn{1}{l}{} & 21 & 25 & 29 & 33 & 3 & 7 & 11 & 15 \\ 
\textbf{21} &  &  &  &  &  &  &  &  & \multicolumn{1}{l}{} & \multicolumn{1}{l}{} & 33 & 7 & 15 & 23 & 31 & 5 & 13 \\ 
\textbf{23} &  &  &  &  &  &  &  &  & \multicolumn{1}{l}{} & \multicolumn{1}{l}{} & \multicolumn{1}{l}{} & 19 & 31 & 9 & 21 & 33 & 11 \\ 
\textbf{25} &  &  &  &  &  &  &  &  & \multicolumn{1}{l}{} & \multicolumn{1}{l}{} & \multicolumn{1}{l}{} & \multicolumn{1}{l}{} & 13 & 29 & 11 & 27 & 9 \\ 
\textbf{27} &  &  &  &  &  &  &  &  & \multicolumn{1}{l}{} & \multicolumn{1}{l}{} & \multicolumn{1}{l}{} & \multicolumn{1}{l}{} & \multicolumn{1}{l}{} & 15 & \textbf{\textit{1}} & 21 & 7 \\ 
\textbf{29} &  &  &  &  &  &  &  &  & \multicolumn{1}{l}{} & \multicolumn{1}{l}{} & \multicolumn{1}{l}{} & \multicolumn{1}{l}{} & \multicolumn{1}{l}{} & \multicolumn{1}{l}{} & 25 & 15 & 5 \\ 
\textbf{31} &  &  &  &  &  &  &  &  & \multicolumn{1}{l}{} & \multicolumn{1}{l}{} & \multicolumn{1}{l}{} & \multicolumn{1}{l}{} & \multicolumn{1}{l}{} & \multicolumn{1}{l}{} & \multicolumn{1}{l}{} & 9 & 3 \\ 
\textbf{33} &  &  &  &  &  &  &  &  & \multicolumn{1}{l}{} & \multicolumn{1}{l}{} & \multicolumn{1}{l}{} & \multicolumn{1}{l}{} & \multicolumn{1}{l}{} & \multicolumn{1}{l}{} & \multicolumn{1}{l}{} & \multicolumn{1}{l}{} & \textbf{\textit{1}} \\ 
\end{tabular}


Thus, the invertible elements of $\left(\mathbb{Z}_{34}^*, \cdot\right)$ are 1,
3, 5, 7, 9, 11, 13, 15, 19, 21, 23, 25, 27, 29, 31, and 33.
\begin{center}
	\begin{tabular}{c|c}
	Member of $\left(\mathbb{Z}_{34}^*, \cdot\right)$ & Inverse, $x^{-1}$ \\
	\hline
	3 & 23 \\
	5 & 7 \\
	% 7 & 5 \\
	9 & 19 \\
	11 & 31 \\
	13 & 21 \\
	15 & 25 \\
	% 19 & 9 \\
	% 21 & 13 \\
	% 23 & 4 \\
	% 25 & 15 \\
	27 & 29 \\
	% 29 & 27 \\
	% 31 & 11 \\
	33 & 33 \\
	% 33 & 33 \\
	\end{tabular}
\end{center}



\section{Extended Euclidean Algorithm}
\begin{equation}
	a=78,\, b=127
\end{equation}
\allowdisplaybreaks
\begin{align*}
	               && \text{gcd}(a,b) & =a*x+b*y\\
	(1)            && \qquad78        & =1*78+0*127     & \\
	(2)            && 127             & =0*78+1*127     & \\
	               &&                 &                 & 127>78\Rightarrow127\text{\,div\,}78=1\\
	(1)            && 78              & =1*78+0*127     & \\
	(2):=(2)-(1)   && 49              & =-1*78+1*127    & \\
	               &&                 &                 & 78>49\Rightarrow78\text{\,div\,}49=1\\
	(1):=(1)-(2)   && 29              & =2*78-1*127     & \\
	(2)            && 49              & =-1*78+1*127    & \\
	               &&                 &                 & 49>29\Rightarrow49\text{\,div\,}29=1\\
	(1)            && 29              & =2*78-1*127     & \\
	(2):=(2)-(1)   && 20              & =-3*78+2*127    & \\
	               &&                 &                 & 29>20\Rightarrow29\text{\,div\,}20=1\\
	(1):=(1)-(2)   && 9               & =5*78-3*127     & \\
	(2)            && 20              & =-3*78+2*127    & \\
	               &&                 &                 & 20>9\Rightarrow20\text{\,div\,}9=2\\
	(1)            && 9               & =5*78-3*127     & \\
	(2):=(2)-2*(1) && 2               & =-13*78+8*127   & \\
	               &&                 &                 & 9>2\Rightarrow9\text{\,div\,}2=8\\
	(1):=(1)-8*(2) && 1               & =57*78-35*127   & \\
	(2)            && 2               & =-13*78+8*127   & \\
	               &&                 &                 & 2>1\Rightarrow2\text{\,div\,}1=2\\
	(1)            && 1               & =57*78-35*127   & \\
	(2):=(2)-2*(1) && 0               & =-127*78+78*127 &
\end{align*}
Thus, the greatest common divisor of 78 and 127, $\text{gcd}(78,127)$, is 1.
\begin{align*}
	\text{gcd}(a,b) &= ( a*x ) + ( b*y ) \\
	\text{gcd}(78,127) &= 1 = ( 78 * 57 ) + ( 127 * -35 ) \\
	&\Rightarrow x=57,\, y=-35 \\
\end{align*}

Since
\begin{equation*}
	ax + by = 1
\end{equation*}
can be reduced modulo $b$ to give
\begin{equation*}
	1 = ax\mod b
\end{equation*}
we also have a value for the multiplicative inverse of $b$ modulo $a$, namely
57 which is the inverse of 78 modulo 127.

\end{document}
