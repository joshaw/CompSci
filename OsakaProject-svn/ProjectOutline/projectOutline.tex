% !TEX TS-program = pdflatex
% !TEX encoding = UTF-8 Unicode
\documentclass[a4paper,10pt]{article}
\usepackage[utf8]{inputenc}

%%% PAGE DIMENSIONS ------------------------------------------------------------
\usepackage[top=3cm]{geometry}
\usepackage[parfill]{parskip}

%%% HEADERS & FOOTERS ----------------------------------------------------------
\usepackage{fancyhdr}
\pagestyle{fancy}
\lhead{}\chead{}\rhead{}
\renewcommand{\headrulewidth}{0pt}
\lfoot{}\cfoot{page \thepage}\rfoot{}

%%% PACKAGES -------------------------------------------------------------------
%\usepackage{morefloats}
\usepackage[hypcap=false,
	font=small,
	labelfont=bf,
	textfont=it]{caption}
\usepackage[pdftex]{graphicx}
\usepackage{booktabs}
\usepackage[strict]{changepage}
\usepackage{array}
\usepackage{enumitem}
\usepackage{longtable}
\usepackage{tabu}
\usepackage{subcaption}
\usepackage{verbatim}
\usepackage{listings}
\usepackage{color}
\usepackage{pdflscape}
\usepackage[percentage]{overpic}
\usepackage{marginfix}
\usepackage[%
	activate={true,nocompatibility},
	final,
	tracking=true,
	kerning=true,
	spacing=true,
	factor=1100,
	stretch=10,
	shrink=10]{microtype}
\microtypecontext{spacing=nonfrench}

%% BIBIOGRAPHY -----------------------------------------------------------------
\usepackage{cite}

%%% ToC (table of contents) APPEARANCE -----------------------------------------
% \usepackage[nottoc,notlof,notlot]{tocbibind}
\usepackage[titles,subfigure]{tocloft}
% \renewcommand{\cftsecfont}{\rmfamily\mdseries\upshape}
% \renewcommand{\cftsecpagefont}{\rmfamily\mdseries\upshape}
% \setcounter{tocdepth}{2}

\setlength\LTleft{0pt}
\setlength\LTright{0pt}

%%% PDF LINKS AND STYLE --------------------------------------------------------
% \usepackage[unicode=true,
% 	bookmarks=true,bookmarksnumbered=true,bookmarksopen=true,
% 	bookmarksopenlevel=2, breaklinks=false,pdfborder={0 0 0},backref=false,
% 	colorlinks=false]{hyperref}
\usepackage[bookmarks=true]{hyperref}
\hypersetup{%
    colorlinks=true,       % false: boxed links; true: colored links
    linkcolor=red,          % color of internal links (change box color with linkbordercolor)
    citecolor=green,        % color of links to bibliography
    filecolor=magenta,      % color of file links
    urlcolor=cyan           % color of external links
}
% \hypersetup{pdftitle={Human Computer Interaction},
% 	pdfauthor={}}

\newcolumntype{Y}{>{\centering\arraybackslash}X}
\newcolumntype{L}{@{}l@{\extracolsep{\fill}}}

\newcommand\marginFig[4][1]{%
	\marginpar{%
		\centering
		\includegraphics[width=#1\marginparwidth]{#2}
		\captionof{figure}{#3}\label{#4}
	}
}
\newcommand{\ts}{\textsuperscript}

%*******************************************************************************
%******************************** END HEADER ***********************************
%*******************************************************************************

\title{Java Software Workshop \\ Project Proposal --- Group Osaka}
\author{%
    Benjamin Crispin \\
    Samuel Farmer \\
    Deedar Fatima \\
    Rowan Stringer \\
    Josh Wainwright
}

\begin{document}

\maketitle

\section{Project Outline}
\label{sec:project_outline}

Our group are proposing an educational tool for use in the classroom which
would allow a group of students to participate in a live quiz competition. A
teacher would be able to input a number of questions which are all either
multiple choice, or have simple numerical answers, along with the answers. The
students are then in a race to answer the question presented as fast as
possible; the next question will be displayed when the set time for the current
question has elapsed.

This program would require the building of a user interface for answering the
questions as well as a means of entering questions and their answers, a server
which handles the questions, serving them, one at a time, to all the students
logged in to it, and a database which holds the questions and answers, as well
as details about the users that have logged in in the past.

\section{System Requirements}
\label{sec:system_requirements}

\begin{itemize}
	\item The system allows Students to log into a user account.
	\item The system allows logged in Students to compete in a many-on-many
		multiple choice/simple-answer quiz.
	\item The system gives a set number of questions per quiz (as per the
		Admin).
	\item The system will allow the Admin to add or remove questions.
	\item The system ensures all clients and the server are synchronised.
	\item The system allows users to log on with past question results.
	\item The system displays a leader board and some feedback about the
		question to the Student.
	\item The system displays a basic statistical information during the quiz
		to the Admin.
	\item The system displays a summary report of statistical information at
		the end of the quiz to the Admin.
	\item The system will log information about the Student's quiz session
		session.
\end{itemize}

\section{Group Roles}
\label{sec:group_roles}

Initial roles are proposed as follows (subject to change throughout duration of
project):
\begin{description}
	\item[Rowan, Deedar] Database implementation, database relationship diagram
		and architecture. Will incorporate defining some parts of API for
		communication between the server and database layers, deciding on
		attributes and table structures and writing documentation for database.
	\item[Sam, Ben, Josh] Server/client model, JDBC connectivity with database.
		Will incorporate client/server socket communication, defining initial
		push/pull structure for communication between clients and server and
		implementing database API via JDBC as defined above.
\end{description}

\section{Planned Timescales}
\label{sec:planned_timescales}

\begin{longtabu}{p{0.33\textwidth}X}
	\toprule
	\textbf{Date to Complete} & \textbf{Task} \\
	\midrule
	Week 1 \newline (25\ts{th} February --- 2\ts{nd} March) &
	\begin{itemize}
		\item Complete database implementation with a basic server/client setup
			that can query the database and write entries to it.
		\item Implement basic JDBC layer for communication between server and
			database.
		\item Several possible prototype diagrams for various GUI interfaces.
	\end{itemize}\\

	Week 2 \newline (3\ts{rd} March --- 9\ts{th} March) &
	\begin{itemize}
		\item Have fully working Admin and Student clients able to communicate
			with server and with database via the server.
		\item Have started implementing GUI for both Admin and Student clients.
	\end{itemize} \\

	Week 3 \newline (10\ts{th} March --- 16\ts{th} March) &
	\begin{itemize}
		\item Finish all elements of GUI including functionality for Admin to
			add and remove questions and Students to answer questions during
			quiz session.
		\item By close of week, fully working quiz system should be
			implemented, incorporating client, server, database and GUI's.
	\end{itemize} \\

	Week 4 \newline (17\ts{th} March --- 23\ts{rd} March) &
	\begin{itemize}
		\item Testing of system.
		\item Final bug fixing, refinements etc.
	\end{itemize} \\

	24\ts{th} March &
	\begin{itemize}
		\item Final hand-in date for project and report.
	\end{itemize} \\
	\bottomrule
\end{longtabu}

\end{document}
